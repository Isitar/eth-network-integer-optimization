\documentclass[12pt, a4]{article}
\usepackage[margin=2cm]{geometry}
\usepackage{parskip}
\usepackage{nameref}
\usepackage{enumitem}
\usepackage{booktabs}
\usepackage{tabularx}
\usepackage{hyperref}
\usepackage[tiny]{titlesec}

\usepackage{amsmath}
\usepackage{amssymb}
\usepackage{amsthm}
\renewcommand\qedsymbol{$\blacksquare$}

\usepackage{fancyhdr}
\usepackage{titling}

\usepackage{pgfplots}
\pgfplotsset{compat=1.16}
\usetikzlibrary{decorations.pathreplacing,positioning, shapes,arrows,chains}


\usepackage{xcolor}
\usepackage{graphicx}
\usepackage{fancyvrb}
\usepackage{listings}
\usepackage{bm}
\usepackage{xcolor}
\usepackage{optidef}

\usepackage{pifont} % for cmark and xmark
\newcommand{\cmark}{\ding{51}}%
\newcommand{\xmark}{\ding{55}}%
\newcommand{\checkedbox}{\rlap{$\square$}{\raisebox{2pt}{\hspace{1pt}\cmark}}}
%\newcommand{\crossedBox}{\rlap{$\square$}{\large\hspace{1pt}\xmark}}


\xdefinecolor{gray}{rgb}{0.4,0.4,0.4}
\xdefinecolor{blue}{RGB}{58,95,205}
\xdefinecolor{darkgreen}{RGB}{0,100,0}

\newcommand{\lightgray}{black!30}

\newcommand{\plotDomain}{-1:8}

\newcommand{\addPlotLDownCoords}[1]{
	\addplot[mark=none, domain=\plotDomain, color=\lightgray,
	decoration={border,segment length=1mm,amplitude=1.5mm,angle=-135},
	postaction={decorate}
	] coordinates {#1};
	\addplot[mark=none, domain=\plotDomain] coordinates {#1};
}

\newcommand{\addPlotLDown}[1]{
	\addplot[mark=none, domain=\plotDomain, color=\lightgray,
	decoration={border,segment length=1mm,amplitude=1.5mm,angle=-135},
	postaction={decorate}
	] {#1};
	\addplot[mark=none, domain=\plotDomain] {#1};
}

\newcommand{\addPlotRUpCoords}[1]{
	\addplot[mark=none, domain=\plotDomain, color=\lightgray,
	decoration={border,segment length=1mm,amplitude=1.5mm,angle=135},
	postaction={decorate}
	] coordinates {#1};
	\addplot[mark=none, domain=\plotDomain] coordinates {#1};
}

\newcommand{\addPlotRUp}[1]{
	\addplot[mark=none, domain=\plotDomain, color=\lightgray,
	decoration={border,segment length=1mm,amplitude=1.5mm,angle=135},
	postaction={decorate}
	] {#1};
	\addplot[mark=none, domain=\plotDomain] {#1};
}

\lstset{% setup listings
	language=R,% set programming language
	basicstyle=\ttfamily\small,% basic font style
	keywordstyle=\color{blue},% keyword style
	commentstyle=\color{gray},% comment style
	numbers=left,% display line numbers on the left side
	numberstyle=\scriptsize,% use small line numbers
	numbersep=10pt,% space between line numbers and code
	tabsize=3,% sizes of tabs
	showstringspaces=false,% do not replace spaces in strings by a certain character
	captionpos=b,% positioning of the caption below
	breaklines=true,% automatic line breaking
	escapeinside={(*}{*)},% escaping to LaTeX
	fancyvrb=true,% verbatim code is typset by listings
	extendedchars=false,% prohibit extended chars (chars of codes 128--255)
	literate={"}{{\texttt{"}}}1{<-}{{$\bm\leftarrow$}}1{<<-}{{$\bm\twoheadleftarrow$}}1
	{~}{{$\bm\sim$}}1{<=}{{$\bm\le$}}1{>=}{{$\bm\ge$}}1{!=}{{$\bm\neq$}}1{^}{{$^{\bm\wedge}$}}1,% item to replace, text, length of chars
	alsoletter={.<-},% becomes a letter
	alsoother={$},% becomes other
	otherkeywords={!=, ~, $, \&, \%/\%, \%*\%, \%\%, <-, <<-, /},% other keywords
	deletekeywords={c}% remove keywords
}

\renewcommand{\arraystretch}{1.2} % more space in tables
\renewcommand\thesubsection{\thesection.\alph{subsection}}
\titleformat{\section}[hang]{\normalfont\bfseries\itshape}{\textup{\thesubsection}}{1em}{}

\titleformat{\subsection}[hang]{\normalsize\itshape}{\textup{\thesubsection}}{1em}{}[]

\newcommand{\norm}[1]{\lvert #1 \rvert}

\newcolumntype{L}{>{$}l<{$}} % math-mode version of "l" column type
\newcolumntype{R}{>{$}r<{$}} % math-mode version of "l" column type
\newcolumntype{C}{>{$}c<{$}} % math-mode version of "l" column type

\author{Pascal Lüscher}
\title{Network \& Integer Optimization – Problem set 2.1}

\makeatletter
\let\mytitle\@title
\makeatother

\pagestyle{fancy}
\fancyhf{}
\rhead{
	\mytitle\\
	\theauthor
}

\rfoot{
	Page: \thepage
}


\newtheorem{definition}{Defintion}

\begin{document}
	\section{Problem 2.1: Night shift planning}
	
	\subsection{Design an efficient algorithm that decides whether there is a feasible assignment, and returns one if it exists. The running time of the algorithm should be in $O(n^c)$ for some constant $c\in\mathbb{Z}\geq0$. Prove that your algorithm is correct and that it obeys the required running time bound.}
	
	\subsubsection{algorithm and construction of the graph}
	First we construct a graph using the information given as following:
	
	\begin{enumerate}
		\item $s$ is the source, 
		\item for each physicians create a node $p_i$ and connect $s$ to all $p_i$ with capacity $\infty$
		\item for each physicians create nodes for each night-block (called $p_{i_{N_l}}$) and connect $p_i$ to $N_l$ with capacity $2$
		\item for each night create a node $a_i$ and connect the corresponding $p_{i_{N_l}}$ to them with capacity $1$ if the physician $p_i$ is available on this night
		\item create a node $t$ (sink) and connect all $a_i$ with capacity $2$
	\end{enumerate}

	An example of such a graph is given below. To get a better visual understanding, the night-blocks are colored.

	\begin{figure}[h]
		\centering
		\scalebox{0.4}{
		\begin{tikzpicture}[x=2.5cm, y=1.5cm, scale=1]
			\begin{scope}[every node/.style={circle,thick,draw, minimum size=8mm}]
				\node (s) at (0,0) {$s$};
				\node (p1) at (1,2) {$p_1$};
				\node (p2) at (1,0) {$p_2$};
				\node (p3) at (1,-2) {$p_3$};			
							
				\node (p1n1)[blue] at (2,4) {$p_{1_{n_1}}$};
				\node (p1n2)[darkgreen] at (2,3) {$p_{1_{n_2}}$};
				\node (p1n3)[purple] at (2,2) {$p_{1_{n_3}}$};
				
				\node (p2n1)[blue] at (2.5,1) {$p_{2_{n_1}}$};
				\node (p2n2)[darkgreen] at (2.5,0) {$p_{2_{n_2}}$};
				\node (p2n3)[purple] at (2.5,-1) {$p_{2_{n_3}}$};
				
				
				\node (p3n1)[blue] at (2,-2) {$p_{3_{n_1}}$};
				\node (p3n2)[darkgreen] at (2,-3) {$p_{3_{n_2}}$};
				\node (p3n3)[purple] at (2,-4) {$p_{3_{n_3}}$};
				
				
				\node (a1)[blue] at (4,3) {$a_1$};
				\node (a2)[blue] at (4,2) {$a_2$};
				\node (a3)[darkgreen] at (4,1) {$a_3$};
				\node (a4)[darkgreen] at (4,0) {$a_4$};
				\node (a5)[purple] at (4,-1) {$a_5$};
				\node (a6)[purple] at (4,-2) {$a_6$};
				
				\node (t) at (5,0) {$t$};
			\end{scope}
			\begin{scope}[every path/.style={thick,}]
				\path (s)  
					edge node[right]{$\infty$} (p1)
					edge node[below]{$\infty$} (p2)
					edge node[right]{$\infty$} (p3)
					;
				\path (p1)
					edge node [right] {$2$} (p1n1)
					edge node [right] {$2$} (p1n2)
					edge node [right] {$2$} (p1n3)
					;
				\path (p2)
					edge node [right] {$2$} (p2n1)
					edge node [right] {$2$} (p2n2)
					edge node [right] {$2$} (p2n3)
					;
				\path (p3)
					edge node [right] {$2$} (p3n1)
					edge node [right] {$2$} (p3n2)
					edge node [right] {$2$} (p3n3)
					;	
					
	
				\begin{scope}[every path/.style={blue}]
					\path (p1n1)
						edge node [right] {$1$} (a1)
						edge node [right] {$1$} (a2)
						;
					\path (p2n1)
						edge node [right] {$0$} (a1)
						edge node [right] {$1$} (a2)
						;
					\path (p3n1)
						edge node [right] {$1$} (a1)
						edge node [right] {$1$} (a2)
						;
				\end{scope}	
						
						
				\begin{scope}[every path/.style={darkgreen}]
					\path (p1n2)
						edge node [right] {$1$} (a3)
						edge node [right] {$0$} (a4)
						;
					\path (p2n2)
						edge node [right] {$1$} (a3)
						edge node [right] {$1$} (a4)
						;
					\path (p3n2)
						edge node [right] {$1$} (a3)
						edge node [right] {$1$} (a4)
						;
				\end{scope}	
					
					
				\begin{scope}[every path/.style={purple}]
					\path (p1n3)
						edge node [right] {$1$} (a5)
						edge node [right] {$1$} (a6)
						;
					\path (p2n3)
						edge node [right] {$1$} (a5)
						edge node [right] {$1$} (a6)
						;
					\path (p3n3)
						edge node [right] {$1$} (a5)
						edge node [right] {$0$} (a6)
						;
				\end{scope}
					
				\path (a1) edge node [right] {$2$} (t);
				\path (a2) edge node [right] {$2$} (t);
				\path (a3) edge node [right] {$2$} (t);
				\path (a4) edge node [right] {$2$} (t);
				\path (a5) edge node [right] {$2$} (t);
				\path (a6) edge node [right] {$2$} (t);
				
			\end{scope}
		\end{tikzpicture}
		}
	\end{figure}

	We then apply the max s-t flow algorithm. 
	\subsubsection{runtime analysis}	
	Running time analysis for the graph construction steps:
	\begin{enumerate}\itemsep-0.5em 
		\item $O(1)$ 
		\item $O(|P|)$
		\item $O(|P|*l)$ ($l$ = number of night-blocks)
		\item $O(|N|*l)$
		\item $O(|N|)$
	\end{enumerate}
	The running time for the Edmonds-Karp s-t algorithm is $O(nm^2)$. We have $|P| + |P|\cdot l + |N| + 2$ nodes and 
	$|P| + |P|\cdot l + l\cdot |N| + |N|$ edges.
	Therefore our algorithm runs in polinomial time.
	
	\subsubsection{proof of correctness}
	
	
\end{document}